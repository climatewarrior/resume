\documentclass[margin,line]{res}

\setlength{\topmargin}{-.5in}

\addtolength{\textheight}{1.75in}

\oddsidemargin -.5in
\evensidemargin -.5in
\textwidth=6.0in


\linespread{0.9}


\begin{document}

\name{\LARGE Gabriel J. P\'erez Irizarry | Computer Science Graduate Student\vspace*{.1in} }


\begin{resume}
\section{\sc Contact Information}
\vspace{.05in}
gabriel.perez@gatech.edu \\
(787) 529-1050       



\section{\sc Objective}
							To expand my software development skills and experience in an innovative and exciting company or start-up through an intership or co-op program.

\section{\sc Education}

{\bf Georgia Institue of Technology \hfill {\bf August 2012 - Present } }
\vspace{-.01cm}
\begin{tabular}{@{}p{2in}p{2in}p{2in}}
 {\bf Major:}  M.S. Computer Science & {\bf Specialization:} Social Computing & {\bf Expected Graduation Date:} May, 2014 \\            
\end{tabular}

{\bf University of Puerto Rico | {\em Mayag\"uez } \hfill {\bf 2006-2012 } }
\vspace{-.01cm}
\begin{tabular}{@{}p{2in}p{2in}p{2in}}
 {\bf Major:} Computer Engineering           & {\bf Graduation Date:} May, 2012  & {\bf GPA:} 3.32/4.00  \\            
\end{tabular}


\section{\sc Skills}
\begin{tabular}{@{}p{2in}p{2in}p{1.7in}}
{\bf Systems Development}             & {\bf Web Development}  & {\bf Project Direction} \\            
Talent for working with large existing code bases and development of complex software systems. & Ability to create, manage and enhance interactive and secure web applications. & Proven ability to lead and manage a wide variety of design and development projects.   
\end{tabular}

\section{\sc Technical}
\begin{tabular}{@{}p{2in}p{2in}p{2in}}
 Java           & AVR Assembly  & GNU/Linux  \\            
Python   & x86 Assembly & Mac OS X   \\
C and C++ & Emacs & Windows XP/Vista/7 \\
Bash & Eclipse & Android \\
PHP & Git/Hg/Svn  & 
\end{tabular}

\section{\sc Work \\Experience}

{\bf Track All Inc, Caguas PR | {\em Mobile Software Developer and Systems Administrator} \hfill {\bf Summer 2011 \\} }
\vspace{-.01cm}
Developed an Android application capable of reporting potholes on the road, acts of vandalism (e.g. graffiti), illegal trash dumping and other problems that affect cities worldwide. People are able to install this application, take a picture of the item, add notes and other information. Then they can upload the data and view a map with all the items.

{\bf Google Summer of Code - Sunlight Foundation | {\em GSoC Student} \hfill {\bf Summer 2010 \\} }
\vspace{-.01cm}
Worked with the Sunlight Foundation on the 50 States Project as a GSoC student. Google Summer of Code is a program in which Google sponsors students to work full-time on Free and Open Source projects during the summer. The 50 States Project wants to make data available from all of the U.S. states legislatures through a single easy to use API. I worked on the development of several scrapers for some states including Hawaii, Colorado and Oregon.


{\bf IBM Linux Technology Center, Austin TX | {\em Pre-Professional Programmer} \hfill {\bf Summer 2009 \\} }
\vspace{-.01cm}
Worked on enhancing and solving issues related to the installer of a Linux distribution developed by IBM. Some of the enhancements include the ability to create live USBs, CDs and virtual machine images. Worked with low-level Linux components such as the initrd/initramfs.

\section{\sc Research}

{\bf University of Puerto Rico, NSF, CenSSIS  | {\em Library Developer} \hfill {\bf 2011 \\} }
\vspace{-.01cm}
The University of Puerto Rico Mayagüez (UPRM) is developing a high performance, documented, and cross-platform GPU library for hyper-spectral image processing. This library takes advantage of GPUs and the CUDA framework by NVIDIA to drastically improve execution times of some hyper-spectral image processing algorithms. A key challenge in the development of the library is portability. I'm working on the development of the build infrastructure and testing infrastructure. Additionally, I'm involved on the creation of its coding guidelines. Conference Paper: Gabriel J. Pérez-Irizarry, Francisco De-La-Cruz, Miguel Velez-Reyes, Nayda Santiago-Santiago, "Developing a portable GPU library for hyperspectral image processing", appeared SPIE Defense, Security and Sensing conference, Baltimore, March 2012.

\clearpage

{\bf University of Puerto Rico, NSF and PR-LSAMP | {\em Game Developer} \hfill {\bf 2010 \\} }
\vspace{-.01cm}
Worked on the development of a serious 3D game-based learning platform in Java using JMonkeyEngine. The project aims to develop a factory simulation game that will help teach Industrial Engineering concepts. My main responsabilities where the integration of the TWL toolkit library, the implementation of the game grid, the refactoring of some third-party dependency and the design of some components.



{\bf University of Puerto Rico, Lockheed Martin, IAP and PR-LSAMP | {\em Library Developer} \hfill {\bf 2009 \\} }
\vspace{-.01cm}
Worked on the development of a serious 3D game-based learning platform in Java using JMonkeyEngine. The project aims to develop a factory simulation game that will help teach Industrial Engineering concepts. My main responsabilities where the integration of the TWL toolkit library, the implementation of the game grid, the refactoring of some third-party dependency and the design of some components.

{\bf University of Puerto Rico | {\em Developer} \hfill {\bf Spring 2008 \\} }
\vspace{-.01cm}
Under the supervision of Prof. Marko Schütz, I contributed to start implementing a software metrics system for NetBSD. This system combines several existing FOSS software metrics systems. My work consisted of the integration of software metrics tools such as cxref and ncc into the NetBSD build process. This work was done as part of the undergraduate research class of the CS department.

\section{\sc Awards}

{\bf Reto 2.0 2011 Award Winner | {\em Web Developer} \hfill {\bf 2011 \\} }
\vspace{-.01cm}
Reto 2.0 is a competition that is open to all college students and it is sponsored by IBM, HP and Microsoft. The idea is to motivate college students to build rich web 2.0 applications. My team built http://enterar.me which roughly translates to: learn. The goal of the site is to combine the strengths of social media and traditional media into a single view. The web site does this by pulling data from Twitter, Facebook and the El Nuevo Día's API.

\section{\sc Student Organizations}

{\bf Free Culture @ UPRM | {\em Founder and President} \hfill {\bf 2007-Present \\} }
\vspace{-.01cm}
Students for Free Culture (SFC) is a diverse, non-partisan group of students and young people who are working to get their peers involved in the free culture movement. SFC chapters exist at over 40 colleges and universities around the world. SFC has collaborated with Creative Commons, the Electronic Frontier Foundation, Public Knowledge, Downhill Battle, and other free software and media reform groups. I co-founded our local chapter and I have help lead dozens of initiatives at our University including: Ubuntu Install Fests, Open Source Game Nights, Free CD Giveaways and a Petition for Free/Open Books. One of our most recent projects, colegiodemocrati.co, was featured in one of Puerto Rico's most popular newspapers, Primera Hora.

{\bf GPM | {\em Founder and Secretary} \hfill {\bf Fall 2006-Spring 2007 \\} }
\vspace{-.01cm}
Co-founded the Multimedia Productions Group. Served as Secretary and worked on the creation of 3D cutscenes for the Ruminix video game project. Also, I prepared various tutorials on how to use Blender for 3D modelling and animation.

\section{\sc School Projects}

{\bf Capstone | {\em Student} \hfill {\bf Fall 2011 \\} }
\vspace{-.01cm}
The Boardcaster is an electronic chess board with an integrated chess engine. The board records chess games and broadcasts them live on the Internet through WiFi. Our system also has the unique feature of illuminating valid moves for player when a piece is raised with lights located throughout each square on the board. I worked on building the LED display system, including hardware and software, and on the WiFi communication. Also I contributed with the PCB verification and development.

{\bf Microprocessor Interfacing | {\em Student} \hfill {\bf Fall 2010 \\} }
\vspace{-.01cm}
Our goal was to use the Arducopter platform, a quadcopter based on Arduino, to create an automatic power-line surveying tool. My main contribution to the project was to get over-the-air serial communication working correctly and reliably between the Arducopter and the ArduRC controller. Additionally, I was involved in the air-worthiness tests performed on the aircraft and developed a GUI application to process the data acquired during missions for use with Google Earth.

\section{\sc More}
Detailed  descriptions, pictures and videos of my work: http://gabrieljperez.nfshost.com/    \\

\end{resume}
\end{document}




